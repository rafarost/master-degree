
%% bare_jrnl.tex
%% V1.3
%% 2007/01/11
%% by Michael Shell
%% see http://www.michaelshell.org/
%% for current contact information.
%%
%% This is a skeleton file demonstrating the use of IEEEtran.cls
%% (requires IEEEtran.cls version 1.7 or later) with an IEEE journal paper.
%%
%% Support sites:
%% http://www.michaelshell.org/tex/ieeetran/
%% http://www.ctan.org/tex-archive/macros/latex/contrib/IEEEtran/
%% and
%% http://www.ieee.org/



% *** Authors should verify (and, if needed, correct) their LaTeX system  ***
% *** with the testflow diagnostic prior to trusting their LaTeX platform ***
% *** with production work. IEEE's font choices can trigger bugs that do  ***
% *** not appear when using other class files.                            ***
% The testflow support page is at:
% http://www.michaelshell.org/tex/testflow/


%%*************************************************************************
%% Legal Notice:
%% This code is offered as-is without any warranty either expressed or
%% implied; without even the implied warranty of MERCHANTABILITY or
%% FITNESS FOR A PARTICULAR PURPOSE!
%% User assumes all risk.
%% In no event shall IEEE or any contributor to this code be liable for
%% any damages or losses, including, but not limited to, incidental,
%% consequential, or any other damages, resulting from the use or misuse
%% of any information contained here.
%%
%% All comments are the opinions of their respective authors and are not
%% necessarily endorsed by the IEEE.
%%
%% This work is distributed under the LaTeX Project Public License (LPPL)
%% ( http://www.latex-project.org/ ) version 1.3, and may be freely used,
%% distributed and modified. A copy of the LPPL, version 1.3, is included
%% in the base LaTeX documentation of all distributions of LaTeX released
%% 2003/12/01 or later.
%% Retain all contribution notices and credits.
%% ** Modified files should be clearly indicated as such, including  **
%% ** renaming them and changing author support contact information. **
%%
%% File list of work: IEEEtran.cls, IEEEtran_HOWTO.pdf, bare_adv.tex,
%%                    bare_conf.tex, bare_jrnl.tex, bare_jrnl_compsoc.tex
%%*************************************************************************

% Note that the a4paper option is mainly intended so that authors in
% countries using A4 can easily print to A4 and see how their papers will
% look in print - the typesetting of the document will not typically be
% affected with changes in paper size (but the bottom and side margins will).
% Use the testflow package mentioned above to verify correct handling of
% both paper sizes by the user's LaTeX system.
%
% Also note that the "draftcls" or "draftclsnofoot", not "draft", option
% should be used if it is desired that the figures are to be displayed in
% draft mode.
%
\documentclass[journal]{IEEEtran}
\usepackage{graphicx}
\usepackage{listings}

% Some very useful LaTeX packages include:
% (uncomment the ones you want to load)


% *** MISC UTILITY PACKAGES ***
%
%\usepackage{ifpdf}
% Heiko Oberdiek's ifpdf.sty is very useful if you need conditional
% compilation based on whether the output is pdf or dvi.
% usage:
% \ifpdf
%   % pdf code
% \else
%   % dvi code
% \fi
% The latest version of ifpdf.sty can be obtained from:
% http://www.ctan.org/tex-archive/macros/latex/contrib/oberdiek/
% Also, note that IEEEtran.cls V1.7 and later provides a builtin
% \ifCLASSINFOpdf conditional that works the same way.
% When switching from latex to pdflatex and vice-versa, the compiler may
% have to be run twice to clear warning/error messages.






% *** CITATION PACKAGES ***
%
%\usepackage{cite}
% cite.sty was written by Donald Arseneau
% V1.6 and later of IEEEtran pre-defines the format of the cite.sty package
% \cite{} output to follow that of IEEE. Loading the cite package will
% result in citation numbers being automatically sorted and properly
% "compressed/ranged". e.g., [1], [9], [2], [7], [5], [6] without using
% cite.sty will become [1], [2], [5]--[7], [9] using cite.sty. cite.sty's
% \cite will automatically add leading space, if needed. Use cite.sty's
% noadjust option (cite.sty V3.8 and later) if you want to turn this off.
% cite.sty is already installed on most LaTeX systems. Be sure and use
% version 4.0 (2003-05-27) and later if using hyperref.sty. cite.sty does
% not currently provide for hyperlinked citations.
% The latest version can be obtained at:
% http://www.ctan.org/tex-archive/macros/latex/contrib/cite/
% The documentation is contained in the cite.sty file itself.






% *** GRAPHICS RELATED PACKAGES ***
%
\ifCLASSINFOpdf
  % \usepackage[pdftex]{graphicx}
  % declare the path(s) where your graphic files are
  % \graphicspath{{../pdf/}{../jpeg/}}
  % and their extensions so you won't have to specify these with
  % every instance of \includegraphics
  % \DeclareGraphicsExtensions{.pdf,.jpeg,.png}
\else
  % or other class option (dvipsone, dvipdf, if not using dvips). graphicx
  % will default to the driver specified in the system graphics.cfg if no
  % driver is specified.
  % \usepackage[dvips]{graphicx}
  % declare the path(s) where your graphic files are
  % \graphicspath{{../eps/}}
  % and their extensions so you won't have to specify these with
  % every instance of \includegraphics
  % \DeclareGraphicsExtensions{.eps}
\fi
% graphicx was written by David Carlisle and Sebastian Rahtz. It is
% required if you want graphics, photos, etc. graphicx.sty is already
% installed on most LaTeX systems. The latest version and documentation can
% be obtained at:
% http://www.ctan.org/tex-archive/macros/latex/required/graphics/
% Another good source of documentation is "Using Imported Graphics in
% LaTeX2e" by Keith Reckdahl which can be found as epslatex.ps or
% epslatex.pdf at: http://www.ctan.org/tex-archive/info/
%
% latex, and pdflatex in dvi mode, support graphics in encapsulated
% postscript (.eps) format. pdflatex in pdf mode supports graphics
% in .pdf, .jpeg, .png and .mps (metapost) formats. Users should ensure
% that all non-photo figures use a vector format (.eps, .pdf, .mps) and
% not a bitmapped formats (.jpeg, .png). IEEE frowns on bitmapped formats
% which can result in "jaggedy"/blurry rendering of lines and letters as
% well as large increases in file sizes.
%
% You can find documentation about the pdfTeX application at:
% http://www.tug.org/applications/pdftex





% *** MATH PACKAGES ***
%
%\usepackage[cmex10]{amsmath}
% A popular package from the American Mathematical Society that provides
% many useful and powerful commands for dealing with mathematics. If using
% it, be sure to load this package with the cmex10 option to ensure that
% only type 1 fonts will utilized at all point sizes. Without this option,
% it is possible that some math symbols, particularly those within
% footnotes, will be rendered in bitmap form which will result in a
% document that can not be IEEE Xplore compliant!
%
% Also, note that the amsmath package sets \interdisplaylinepenalty to 10000
% thus preventing page breaks from occurring within multiline equations. Use:
%\interdisplaylinepenalty=2500
% after loading amsmath to restore such page breaks as IEEEtran.cls normally
% does. amsmath.sty is already installed on most LaTeX systems. The latest
% version and documentation can be obtained at:
% http://www.ctan.org/tex-archive/macros/latex/required/amslatex/math/





% *** SPECIALIZED LIST PACKAGES ***
%
%\usepackage{algorithmic}
% algorithmic.sty was written by Peter Williams and Rogerio Brito.
% This package provides an algorithmic environment fo describing algorithms.
% You can use the algorithmic environment in-text or within a figure
% environment to provide for a floating algorithm. Do NOT use the algorithm
% floating environment provided by algorithm.sty (by the same authors) or
% algorithm2e.sty (by Christophe Fiorio) as IEEE does not use dedicated
% algorithm float types and packages that provide these will not provide
% correct IEEE style captions. The latest version and documentation of
% algorithmic.sty can be obtained at:
% http://www.ctan.org/tex-archive/macros/latex/contrib/algorithms/
% There is also a support site at:
% http://algorithms.berlios.de/index.html
% Also of interest may be the (relatively newer and more customizable)
% algorithmicx.sty package by Szasz Janos:
% http://www.ctan.org/tex-archive/macros/latex/contrib/algorithmicx/




% *** ALIGNMENT PACKAGES ***
%
%\usepackage{array}
% Frank Mittelbach's and David Carlisle's array.sty patches and improves
% the standard LaTeX2e array and tabular environments to provide better
% appearance and additional user controls. As the default LaTeX2e table
% generation code is lacking to the point of almost being broken with
% respect to the quality of the end results, all users are strongly
% advised to use an enhanced (at the very least that provided by array.sty)
% set of table tools. array.sty is already installed on most systems. The
% latest version and documentation can be obtained at:
% http://www.ctan.org/tex-archive/macros/latex/required/tools/


%\usepackage{mdwmath}
%\usepackage{mdwtab}
% Also highly recommended is Mark Wooding's extremely powerful MDW tools,
% especially mdwmath.sty and mdwtab.sty which are used to format equations
% and tables, respectively. The MDWtools set is already installed on most
% LaTeX systems. The lastest version and documentation is available at:
% http://www.ctan.org/tex-archive/macros/latex/contrib/mdwtools/


% IEEEtran contains the IEEEeqnarray family of commands that can be used to
% generate multiline equations as well as matrices, tables, etc., of high
% quality.


%\usepackage{eqparbox}
% Also of notable interest is Scott Pakin's eqparbox package for creating
% (automatically sized) equal width boxes - aka "natural width parboxes".
% Available at:
% http://www.ctan.org/tex-archive/macros/latex/contrib/eqparbox/





% *** SUBFIGURE PACKAGES ***
%\usepackage[tight,footnotesize]{subfigure}
% subfigure.sty was written by Steven Douglas Cochran. This package makes it
% easy to put subfigures in your figures. e.g., "Figure 1a and 1b". For IEEE
% work, it is a good idea to load it with the tight package option to reduce
% the amount of white space around the subfigures. subfigure.sty is already
% installed on most LaTeX systems. The latest version and documentation can
% be obtained at:
% http://www.ctan.org/tex-archive/obsolete/macros/latex/contrib/subfigure/
% subfigure.sty has been superceeded by subfig.sty.



%\usepackage[caption=false]{caption}
%\usepackage[font=footnotesize]{subfig}
% subfig.sty, also written by Steven Douglas Cochran, is the modern
% replacement for subfigure.sty. However, subfig.sty requires and
% automatically loads Axel Sommerfeldt's caption.sty which will override
% IEEEtran.cls handling of captions and this will result in nonIEEE style
% figure/table captions. To prevent this problem, be sure and preload
% caption.sty with its "caption=false" package option. This is will preserve
% IEEEtran.cls handing of captions. Version 1.3 (2005/06/28) and later
% (recommended due to many improvements over 1.2) of subfig.sty supports
% the caption=false option directly:
%\usepackage[caption=false,font=footnotesize]{subfig}
%
% The latest version and documentation can be obtained at:
% http://www.ctan.org/tex-archive/macros/latex/contrib/subfig/
% The latest version and documentation of caption.sty can be obtained at:
% http://www.ctan.org/tex-archive/macros/latex/contrib/caption/




% *** FLOAT PACKAGES ***
%
%\usepackage{fixltx2e}
% fixltx2e, the successor to the earlier fix2col.sty, was written by
% Frank Mittelbach and David Carlisle. This package corrects a few problems
% in the LaTeX2e kernel, the most notable of which is that in current
% LaTeX2e releases, the ordering of single and double column floats is not
% guaranteed to be preserved. Thus, an unpatched LaTeX2e can allow a
% single column figure to be placed prior to an earlier double column
% figure. The latest version and documentation can be found at:
% http://www.ctan.org/tex-archive/macros/latex/base/



%\usepackage{stfloats}
% stfloats.sty was written by Sigitas Tolusis. This package gives LaTeX2e
% the ability to do double column floats at the bottom of the page as well
% as the top. (e.g., "\begin{figure*}[!b]" is not normally possible in
% LaTeX2e). It also provides a command:
%\fnbelowfloat
% to enable the placement of footnotes below bottom floats (the standard
% LaTeX2e kernel puts them above bottom floats). This is an invasive package
% which rewrites many portions of the LaTeX2e float routines. It may not work
% with other packages that modify the LaTeX2e float routines. The latest
% version and documentation can be obtained at:
% http://www.ctan.org/tex-archive/macros/latex/contrib/sttools/
% Documentation is contained in the stfloats.sty comments as well as in the
% presfull.pdf file. Do not use the stfloats baselinefloat ability as IEEE
% does not allow \baselineskip to stretch. Authors submitting work to the
% IEEE should note that IEEE rarely uses double column equations and
% that authors should try to avoid such use. Do not be tempted to use the
% cuted.sty or midfloat.sty packages (also by Sigitas Tolusis) as IEEE does
% not format its papers in such ways.


%\ifCLASSOPTIONcaptionsoff
%  \usepackage[nomarkers]{endfloat}
% \let\MYoriglatexcaption\caption
% \renewcommand{\caption}[2][\relax]{\MYoriglatexcaption[#2]{#2}}
%\fi
% endfloat.sty was written by James Darrell McCauley and Jeff Goldberg.
% This package may be useful when used in conjunction with IEEEtran.cls'
% captionsoff option. Some IEEE journals/societies require that submissions
% have lists of figures/tables at the end of the paper and that
% figures/tables without any captions are placed on a page by themselves at
% the end of the document. If needed, the draftcls IEEEtran class option or
% \CLASSINPUTbaselinestretch interface can be used to increase the line
% spacing as well. Be sure and use the nomarkers option of endfloat to
% prevent endfloat from "marking" where the figures would have been placed
% in the text. The two hack lines of code above are a slight modification of
% that suggested by in the endfloat docs (section 8.3.1) to ensure that
% the full captions always appear in the list of figures/tables - even if
% the user used the short optional argument of \caption[]{}.
% IEEE papers do not typically make use of \caption[]'s optional argument,
% so this should not be an issue. A similar trick can be used to disable
% captions of packages such as subfig.sty that lack options to turn off
% the subcaptions:
% For subfig.sty:
% \let\MYorigsubfloat\subfloat
% \renewcommand{\subfloat}[2][\relax]{\MYorigsubfloat[]{#2}}
% For subfigure.sty:
% \let\MYorigsubfigure\subfigure
% \renewcommand{\subfigure}[2][\relax]{\MYorigsubfigure[]{#2}}
% However, the above trick will not work if both optional arguments of
% the \subfloat/subfig command are used. Furthermore, there needs to be a
% description of each subfigure *somewhere* and endfloat does not add
% subfigure captions to its list of figures. Thus, the best approach is to
% avoid the use of subfigure captions (many IEEE journals avoid them anyway)
% and instead reference/explain all the subfigures within the main caption.
% The latest version of endfloat.sty and its documentation can obtained at:
% http://www.ctan.org/tex-archive/macros/latex/contrib/endfloat/
%
% The IEEEtran \ifCLASSOPTIONcaptionsoff conditional can also be used
% later in the document, say, to conditionally put the References on a
% page by themselves.





% *** PDF, URL AND HYPERLINK PACKAGES ***
%
%\usepackage{url}
% url.sty was written by Donald Arseneau. It provides better support for
% handling and breaking URLs. url.sty is already installed on most LaTeX
% systems. The latest version can be obtained at:
% http://www.ctan.org/tex-archive/macros/latex/contrib/misc/
% Read the url.sty source comments for usage information. Basically,
% \url{my_url_here}.





% *** Do not adjust lengths that control margins, column widths, etc. ***
% *** Do not use packages that alter fonts (such as pslatex).         ***
% There should be no need to do such things with IEEEtran.cls V1.6 and later.
% (Unless specifically asked to do so by the journal or conference you plan
% to submit to, of course. )


% correct bad hyphenation here
\hyphenation{op-tical net-works semi-conduc-tor}


\begin{document}
%
% paper title
% can use linebreaks \\ within to get better formatting as desired
\title{Coursework 2 - One Time Pad Attack}
%
%
% author names and IEEE memberships
% note positions of commas and nonbreaking spaces ( ~ ) LaTeX will not break
% a structure at a ~ so this keeps an author's name from being broken across
% two lines.
% use \thanks{} to gain access to the first footnote area
% a separate \thanks must be used for each paragraph as LaTeX2e's \thanks
% was not built to handle multiple paragraphs
%

\author{Rafael~Rost}

% note the % following the last \IEEEmembership and also \thanks -
% these prevent an unwanted space from occurring between the last author name
% and the end of the author line. i.e., if you had this:
%
% \author{....lastname \thanks{...} \thanks{...} }
%                     ^------------^------------^----Do not want these spaces!
%
% a space would be appended to the last name and could cause every name on that
% line to be shifted left slightly. This is one of those "LaTeX things". For
% instance, "\textbf{A} \textbf{B}" will typeset as "A B" not "AB". To get
% "AB" then you have to do: "\textbf{A}\textbf{B}"
% \thanks is no different in this regard, so shield the last } of each \thanks
% that ends a line with a % and do not let a space in before the next \thanks.
% Spaces after \IEEEmembership other than the last one are OK (and needed) as
% you are supposed to have spaces between the names. For what it is worth,
% this is a minor point as most people would not even notice if the said evil
% space somehow managed to creep in.



% The paper headers
% The only time the second header will appear is for the odd numbered pages
% after the title page when using the twoside option.
%
% *** Note that you probably will NOT want to include the author's ***
% *** name in the headers of peer review papers.                   ***
% You can use \ifCLASSOPTIONpeerreview for conditional compilation here if
% you desire.




% If you want to put a publisher's ID mark on the page you can do it like
% this:
%\IEEEpubid{0000--0000/00\$00.00~\copyright~2007 IEEE}
% Remember, if you use this you must call \IEEEpubidadjcol in the second
% column for its text to clear the IEEEpubid mark.



% use for special paper notices
%\IEEEspecialpapernotice{(Invited Paper)}




% make the title area
\maketitle


\begin{abstract}
%\boldmath
In order to complete a coursework of master degree cryptography classes, in this report I have presented how to decrypt a series of messages that was encrypted using One Time Pad cipher, by taking the advantage of key reutilization. As part of the cryptanalysis work, a tool was developed using Ruby language. The tool did not decrypted all the messages, but was very helpful and saved me a lot of hard working. Source-code is available on my GitHub account.
\end{abstract}
% IEEEtran.cls defaults to using nonbold math in the Abstract.
% This preserves the distinction between vectors and scalars. However,
% if the journal you are submitting to favors bold math in the abstract,
% then you can use LaTeX's standard command \boldmath at the very start
% of the abstract to achieve this. Many IEEE journals frown on math
% in the abstract anyway.

% Note that keywords are not normally used for peerreview papers.
\begin{IEEEkeywords}
One Time Pad, Ruby, Cryptography.
\end{IEEEkeywords}

% For peer review papers, you can put extra information on the cover
% page as needed:
% \ifCLASSOPTIONpeerreview
% \begin{center} \bfseries EDICS Category: 3-BBND \end{center}
% \fi
%
% For peerreview papers, this IEEEtran command inserts a page break and
% creates the second title. It will be ignored for other modes.
\IEEEpeerreviewmaketitle



\section{Introduction}
In theory, One Time Pad is a Stream Cipher that is consider perfectly secure. Basically, this cipher encrypts messages using a unique key that has the same size of the message. In other words, each message byte is encrypted with relative key byte.

However, this cipher is only secure if is properly used. If for some reason, the same key is used to encrypt more than one message, there is a big chance to perform a cryptanalysis and reveal messages content.

% needed in second column of first page if using \IEEEpubid
%\IEEEpubidadjcol

% An example of a floating figure using the graphicx package.
% Note that \label must occur AFTER (or within) \caption.
% For figures, \caption should occur after the \includegraphics.
% Note that IEEEtran v1.7 and later has special internal code that
% is designed to preserve the operation of \label within \caption
% even when the captionsoff option is in effect. However, because
% of issues like this, it may be the safest practice to put all your
% \label just after \caption rather than within \caption{}.
%
% Reminder: the "draftcls" or "draftclsnofoot", not "draft", class
% option should be used if it is desired that the figures are to be
% displayed while in draft mode.
%
%\begin{figure}[!t]
%\centering
%\includegraphics[width=2.5in]{myfigure}
% where an .eps filename suffix will be assumed under latex,
% and a .pdf suffix will be assumed for pdflatex; or what has been declared
% via \DeclareGraphicsExtensions.
%\caption{Simulation Results}
%\label{fig_sim}
%\end{figure}

% Note that IEEE typically puts floats only at the top, even when this
% results in a large percentage of a column being occupied by floats.


% An example of a double column floating figure using two subfigures.
% (The subfig.sty package must be loaded for this to work.)
% The subfigure \label commands are set within each subfloat command, the
% \label for the overall figure must come after \caption.
% \hfil must be used as a separator to get equal spacing.
% The subfigure.sty package works much the same way, except \subfigure is
% used instead of \subfloat.
%
%\begin{figure*}[!t]
%\centerline{\subfloat[Case I]\includegraphics[width=2.5in]{subfigcase1}%
%\label{fig_first_case}}
%\hfil
%\subfloat[Case II]{\includegraphics[width=2.5in]{subfigcase2}%
%\label{fig_second_case}}}
%\caption{Simulation results}
%\label{fig_sim}
%\end{figure*}
%
% Note that often IEEE papers with subfigures do not employ subfigure
% captions (using the optional argument to \subfloat), but instead will
% reference/describe all of them (a), (b), etc., within the main caption.


% An example of a floating table. Note that, for IEEE style tables, the
% \caption command should come BEFORE the table. Table text will default to
% \footnotesize as IEEE normally uses this smaller font for tables.
% The \label must come after \caption as always.
%
%\begin{table}[!t]
%% increase table row spacing, adjust to taste
%\renewcommand{\arraystretch}{1.3}
% if using array.sty, it might be a good idea to tweak the value of
% \extrarowheight as needed to properly center the text within the cells
%\caption{An Example of a Table}
%\label{table_example}
%\centering
%% Some packages, such as MDW tools, offer better commands for making tables
%% than the plain LaTeX2e tabular which is used here.
%\begin{tabular}{|c||c|}
%\hline
%One & Two\\
%\hline
%Three & Four\\
%\hline
%\end{tabular}
%\end{table}


% Note that IEEE does not put floats in the very first column - or typically
% anywhere on the first page for that matter. Also, in-text middle ("here")
% positioning is not used. Most IEEE journals use top floats exclusively.
% Note that, LaTeX2e, unlike IEEE journals, places footnotes above bottom
% floats. This can be corrected via the \fnbelowfloat command of the
% stfloats package.

\section{One Time Pad}

\subsection{Understanding the Algorithm}
The encrypt/decrypt algorithm behind OTP is very simple. The cipher-text is the result of a XOR operation between message and key, like we can see in this example:

\begin{itemize}
	\item Message     : 0 1 0 1 1 0
	\item Key         : 1 1 1 0 0 1
	\item Cipher-text : 1 0 1 1 1 1
\end{itemize}

One of the advantages of this cipher, is that it is really fast to decrypt/encrypt messages. But, the big problem is that the key must have the same size of the message, and sometimes this turns OTP an impractical cipher.

\subsection{XOR Operation}
Since the main OTP cipher operation is the XOR, let’s try to understand a little bit more about this operation. XOR operation basically returns true whenever the two values differs and false when they are equals. We can see this in the XOR truth table:

\begin{itemize}
	\item 0 \textasciicircum{} 0 = 0
	\item 0 \textasciicircum{} 1 = 1
	\item 1 \textasciicircum{} 0 = 1
	\item 1 \textasciicircum{} 1 = 0
\end{itemize}

One important thing that we can notice on this truth table, is that is possible to recover the values that were xored together. For instance, xoring 1 with 1 give us 0 as result. Xoring the result (0) with 1, give us 1 again. This show us that xoring the cipher-text with the key, will give us the plaintext, and it’ possible to check this by inverting the truth table:

\begin{itemize}
	\item 0 = 0 \textasciicircum{} 0
	\item 0 = 1 \textasciicircum{} 1
	\item 1 = 0 \textasciicircum{} 1
	\item 1 = 1 \textasciicircum{} 0
\end{itemize}

\section{Cryptanalysis}

The cryptanalysis process was done in two parts. The first one, was performed in an automated way, using a tool that was coded using Ruby, and it decrypted almost entire cipher-text. The second part was done by hand, and it was just fill or correct some words into the text, by using obvious guesses. But, in order to confirm the guesses, I ran the tool again hard coding the key, and this showed the entire plaintext and key.

As part of this coursework, we also received a hint about the cryptanalysis process: “Consider what happens when something is xored with space (hex 32)”. So, let’s check that, by playing with some Ruby code.

\lstinputlisting[language=Ruby, firstline=141, lastline=146]{one-time-pad.rb}
\begin{itemize}
	\item Result: ABCDEFGHIJKLMNOPQRSTUVWXYZ
\end{itemize}

\lstinputlisting[language=Ruby, firstline=148, lastline=153]{one-time-pad.rb}
\begin{itemize}
	\item Result: abcdefghijklmnopqrstuvwxyz
\end{itemize}

So, basically when you XOR an letter with a space, the letter will swap case, which means, an “A” becames an “a” and vice-versa. Also, let’s check what happens when we XOR a space with another space.

\lstinputlisting[language=Ruby, firstline=155, lastline=156]{one-time-pad.rb}
Result: ' '

Xoring two spaces, gave us another space. With this hints on mind and also knowing a little bit about the XOR operation, we have now food enough to start performing the cryptanalysis process. The next session of this paper will describe how the cryptanalysis was implemented on the ruby decrypt tool.


\subsection{Xoring cipher texts together}
At this time, all that we know is that a list of messages was encrypted using the same “OTP” key. So, let’s now imagine this scenario:

\begin{itemize}
	\item Message1: hello
	\item Message2: devil
	\item Key     : abcde
\end{itemize}

In this case, the first cipher-text will be:

\begin{itemize}
	\item (h \textasciicircum{} a) (e \textasciicircum{} b) (l \textasciicircum{} c) (l \textasciicircum{} d) (o \textasciicircum{} e)
\end{itemize}

And the second one will be:

\begin{itemize}
	\item (d \textasciicircum{} a) (e \textasciicircum{} b) (v \textasciicircum{} c) (i \textasciicircum{} d) (l \textasciicircum{} e)
\end{itemize}

Let’s see what happens when we XOR this two cipher-texts together.

\begin{itemize}
	\item (h \textasciicircum{} a) \textasciicircum{} (d \textasciicircum{} a) = (h \textasciicircum{} d) \textasciicircum{} (a \textasciicircum{} a) 
\end{itemize}

Now, let’s play with ruby again, and see what happens when we XOR a value with itself.

\lstinputlisting[language=Ruby, firstline=158, lastline=159]{one-time-pad.rb}
\begin{itemize}
	\item Result:
\end{itemize}

We got NULL! So, we can say that our equation could looks like this:

\begin{itemize}
	\item (h \textasciicircum{} d) \textasciicircum{} NULL
\end{itemize}

Finally, we can check the result of a XOR between a value and NULL:

\lstinputlisting[language=Ruby, firstline=161, lastline=162]{one-time-pad.rb}
\begin{itemize}
	\item Result: B
\end{itemize}

Like we can see, we got the value itself as a result, which means that the final result of our equation will be:

\begin{itemize}
	\item (h \textasciicircum{} d)
\end{itemize}

Doing the same with all the message letters, we will have this:

\begin{itemize}
	\item (h \textasciicircum{} d) (e \textasciicircum{} e) (l \textasciicircum{} v) (l \textasciicircum{} i) (o \textasciicircum{} l)
\end{itemize}

Nice! We ended up with a new cipher text, the difference of this one is that we removed the key from it!

We also learned that a XOR between space and letter will swap case the letter. So, we can look at the new cipher text and assume that every character between [a-zA-Z] is the product of a xor between a space and another character. We can also assume that all spaces are the product of two another spaces. 

At this point, by xoring two cipher texts encrypted with the same key, we are able to identify that one of them have a space on a specific position. Let’s now discover which one has that space.

\subsection{Discovering empty spaces}
Supposing that we have this messages A and B, which were encrypted using OTP:

\begin{itemize}
	\item A: “hello my friend”
	\item B: “give me the key”
\end{itemize}

Xoring the cipher texts of this two messages gave us something like this:

\begin{itemize}
	\item * * * *\_ \_ * \_ \_ * * \_ * * *
\end{itemize}

Since we are only interested in the spaces, I represented all another cipher texts characters with “*”. We know that messages A or B have a space on the positions represented by “\_”. Now, let’s introduce a new message to our scenario, and XOR it with message A.

\begin{itemize}
	\item A: “hello my friend”
	\item C: “hey friend”
\end{itemize}

The result of XOR between the cipher texts of this two messages will be:

\begin{itemize}
	\item * * * \_ * \_ * * \_ * * * * *
\end{itemize}

Since we are comparing message A with another two messages, we can consider that every space found on same position in two comparisons, must belong to message A. Look at the result:

\begin{itemize}
	\item * * * * \_ \_ * \_ \_ * * \_ * * (A with B)
	\item * * * \_ * \_ * * \_ * * * * * (A with C)
	\item * * * * * \_ * * \_ * * * * * (common spaces between comparisons)
\end{itemize}

Now we discovered exactly which spaces belongs to message A. Off course, this method is not so precise, because we can have spaces at the same positions at B and C messages, and this can drive us to some false positives. However, the more messages we have, more comparisons we can perform and more precise our result will be. And that’s the case of this coursework. On the ruby tool, I rotated all the messages and did this comparisons between each message with all another messages. The result of this, was that I discovered all the spaces to all the messages.

\subsection{Finding the key}
Knowing all the spaces of all the messages gave us the ability to access the key. We simple XOR the message spaces positions with a space and this will reveal the key. At this point, the automated cryptanalysis was done, and we ended up with this plain text:

m0:    CAESA\_   A\_D   \_IGENERE  \_A\_E  CIPHER\_\_\_HA\_  \_CA\_ B\_\_  E\_SI\_\_   BR\_\_\_\_\_

m1:THIS  IS \_HE  \_ECON\_   STRING\_T\_AT  WAS  \_\_\_RY\_TE\_  \_SI\_\_  O\_E \_\_ME   \_\_\_\_\_

m2:ONE TIME \_AD W\_LL P\_OVIDE A  \_I\_HERTEXT O\_\_\_ A\_TA\_K \_EC\_\_ITY\_

m3:IN THE  C\_IPTO\_RAPH\_  CLASSES\_Y\_U WILL LE\_\_\_ H\_W \_O \_ON\_\_SE \_ND\_\_O DIF\_\_\_\_\_

m4:TO UNDERS\_AND \_ISCR\_TE PROBAB\_L\_TY IS IMP\_\_\_AN\_ T\_ U\_DE\_\_TAN\_ C\_\_PTOGR\_\_\_\_\_

m5:ENGLISH L\_TTER\_ FRE\_UENCY ALL\_W\_ YOU TO B\_\_\_K \_IG\_NE\_E \_\_D C\_ES\_\_ CIPH\_\_\_\_

m6:CRYPTOGRA\_HY I\_ PRE\_ENT IN SE\_E\_AL DIFFER\_\_\_ T\_PE\_ O\_ A\_\_LIC\_TI\_\_S NOW\_\_\_\_\_\_

m7:IF YOU FO\_ND T\_E ON\_ TIME PAD\_ \_HEN YOU H\_\_\_ F\_NI\_HE\_ T\_\_S  \_OU\_\_EWORK\_

m8:Every clo\_d ha\_ a s\_lver lini\_g\_

This is awesome! As you can see, now it’s a piece of cake to guess and fill the missing words inside the messages.

\subsection{Guessing}
Now, that we have a partial key, and also a good idea of the plaintext, let’s perform some guesses in order to discover the entire key. Look at message seven partial plain text: IF YOU FO\_ND T\_E ON\_ TIME PAD\_ \_HEN YOU H\_\_\_ F\_NI\_HE\_ T\_\_S  \_OU\_\_EWORK\_.It looks like we can have a good guess for the entire message: IF YOU FOUND THE ONE TIME PAD THEN YOU HAVE FINISHED THIS COURSEWORK.

Good guess right? First empty letter that we fill was U at position nine. So we XOR the U with the cipher text at the same position, and we got the key, also at the same position. We keep doing that to fill all the empty spaces, and we will have more key bytes. On the ruby tool, I hard coded my guesses like this:

\lstinputlisting[language=Ruby, firstline=174, lastline=174]{one-time-pad.rb}

We are really closer to the end. Now, it's all about keep guessing and running the ruby tool again. At the end, entire plain text will be revealed.

m0:    CAESAR   AND   VIGENERE   ARE  CIPHERS THAT   CAN BE   EASILY   BROKEN.

m1:THIS  IS THE  SECOND   STRING THAT  WAS  ENCRYPTED  USING  ONE TIME    PAD.

m2:ONE TIME PAD WILL PROVIDE A  CIPHERTEXT ONLY ATTACK SECURITY.

m3:IN THE  CRIPTOGRAPHY  CLASSES YOU WILL LEARN HOW TO CONFUSE AND TO DIFFUSE.

m4:TO UNDERSTAND DISCRETE PROBABILITY IS IMPORTANT TO UNDERSTAND CRYPTOGRAPHY.

m5:ENGLISH LETTERS FREQUENCY ALLOWS YOU TO BREAK VIGENERE AND CAESAR CIPHERS.

m6:CRYPTOGRAPHY IS PRESENT IN SEVERAL DIFFERENT TYPES OF APPLICATIONS NOWADAYS.

m7:IF YOU FOUND THE ONE TIME PAD, THEN YOU HAVE FINISHED THIS  COURSEWORK.

m8:Every cloud has a silver lining.

\section{Conclusion}
Even we considering that one time pad has the perfect secrecy, in this coursework we proved that if we perform an inadequate use of the cipher, is completely possible to break it. We just have to understand a little bit how the XOR operation works, and also explore some hints about it. Most part of the cryptanalysis was completely automated. In fact, just with the result that the tool gave to us, is possible to guess and understand the entire text. I just implemented the guess part to make sure that my guesses are right.

% if have a single appendix:
%\appendix[Proof of the Zonklar Equations]
% or
%\appendix  % for no appendix heading
% do not use \section anymore after \appendix, only \section*
% is possibly needed

% use appendices with more than one appendix
% then use \section to start each appendix
% you must declare a \section before using any
% \subsection or using \label (\appendices by itself
% starts a section numbered zero.)
%

% Can use something like this to put references on a page
% by themselves when using endfloat and the captionsoff option.
\ifCLASSOPTIONcaptionsoff
  \newpage
\fi



% trigger a \newpage just before the given reference
% number - used to balance the columns on the last page
% adjust value as needed - may need to be readjusted if
% the document is modified later
%\IEEEtriggeratref{8}
% The "triggered" command can be changed if desired:
%\IEEEtriggercmd{\enlargethispage{-5in}}

% references section

% can use a bibliography generated by BibTeX as a .bbl file
% BibTeX documentation can be easily obtained at:
% http://www.ctan.org/tex-archive/biblio/bibtex/contrib/doc/
% The IEEEtran BibTeX style support page is at:
% http://www.michaelshell.org/tex/ieeetran/bibtex/
%\bibliographystyle{IEEEtran}
% argument is your BibTeX string definitions and bibliography database(s)
%\bibliography{IEEEabrv,../bib/paper}
%
% <OR> manually copy in the resultant .bbl file
% set second argument of \begin to the number of references
% (used to reserve space for the reference number labels box)
%\begin{thebibliography}{1}

%\bibitem{IEEEhowto:kopka}
%H.~Kopka and P.~W. Daly, \emph{A Guide to \LaTeX}, 3rd~ed.\hskip 1em plus
%  0.5em minus 0.4em\relax Harlow, England: Addison-Wesley, 1999.

%\end{thebibliography}

% biography section
%
% If you have an EPS/PDF photo (graphicx package needed) extra braces are
% needed around the contents of the optional argument to biography to prevent
% the LaTeX parser from getting confused when it sees the complicated
% \includegraphics command within an optional argument. (You could create
% your own custom macro containing the \includegraphics command to make things
% simpler here.)
%\begin{biography}[{\includegraphics[width=1in,height=1.25in,clip,keepaspectratio]{mshell}}]{Michael Shell}
% or if you just want to reserve a space for a photo:

\begin{biography}[{\includegraphics[width=1in,height=1.25in,clip,keepaspectratio]{picture}}]{Rafael Rost}
	Software engineer at HP Inc. GitHub account: github.com/rafarost
\end{biography}

% You can push biographies down or up by placing
% a \vfill before or after them. The appropriate
% use of \vfill depends on what kind of text is
% on the last page and whether or not the columns
% are being equalized.

%\vfill

% Can be used to pull up biographies so that the bottom of the last one
% is flush with the other column.
%\enlargethispage{-5in}



% that's all folks
\end{document}
